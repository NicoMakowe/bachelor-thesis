\newcounter{UseCaseNumber}
\setcounter{UseCaseNumber}{1}

\newcommand{\usecaseSmall}[2]{
	\subsection*{Anwendungsfall \arabic{UseCaseNumber}: #1}
	\begin{tabular}{|p{6cm}|p{18.5cm}|}
		\hline
		Name & #1 \\
		\hline
		Kurzbeschreibung & #2 \\
		\hline
	\end{tabular}
	\stepcounter{UseCaseNumber}
}

\newcommand{\usecaseBigA}[6]{%
	\def\usecaseArgA{#1}%
	\def\usecaseArgB{#2}%
	\def\usecaseArgC{#3}%
	\def\usecaseArgD{#4}%
	\def\usecaseArgE{#5}%
	\def\usecaseArgF{#6}%
}

\newcommand{\usecaseBigB}[6]{
	\subsection*{Anwendungsfall \arabic{UseCaseNumber}: \usecaseArgA}
	\begin{tabular}{|p{6cm}|p{18.5cm}|}
		\hline
		Name & \usecaseArgA \\
		\hline
		Kurzbeschreibung & \usecaseArgB \\
		\hline
		fachliches Ziel & \usecaseArgC \\
		\hline
		Akteure & \usecaseArgD \\
		\hline
		Auslöser & \usecaseArgE \\
		\hline
		Vorbedingung & \usecaseArgF \\
		\hline
		Nachbedingung & #1 \\
		\hline
		verwendete Informationen & #2 \\
		\hline
		Ergebnis & #3 \\
		\hline
		Ablauf & #4 \\
		\hline
		Abläufe in Ausnahmefällen & #5 \\
		\hline
		Verweise & #6 \\
		\hline
	\end{tabular}
	\stepcounter{UseCaseNumber}
}