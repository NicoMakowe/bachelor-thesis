\chapter{Maßnahmen zur Absicherung des toten Winkels}
	\section{Totwinkelassistenten}
		Es gibt keine genormte Definition für den Begriff Totwinkelassistent und in der Praxis wird der Begriff teilweise für unterschiedliche Systeme verwendet. Aus diesem Grund wird für diese Arbeit eine eigene Definition zugrunde gelegt.
		
		Ein Totwinkelassistent ist ein Fahrerassistenzsystem, das die toten Winkel links und rechts vom Fahrzeug hinter dem Fahrer überwacht. Sollte sich ein anderer Verkehrsteilnehmer, ein Hindernis oder ein Fußgänger innerhalb dieses Bereichs befinden, wird der Fahrer gewarnt. Die Systeme mancher Fahrzeug-Modelle können in das Fahrverhalten eingreifen, wenn der Fahrer trotz Warnung ein gefährliches Manöver einleitet. Totwinkelassistenten werden im Folgenden in zwei Gruppen eingeteilt. (siehe Abbildung \ref{fig:totwinkelassistenten})
		\begin{figure}
			\centering
			\includegraphics[width=1\linewidth]{images/Totwinkelassistenten}
			\caption{Einteilung von Totwinkelassistenten}
			\label{fig:totwinkelassistenten}
		\end{figure}
		
		\begin{enumerate}
			\item Die erste Gruppe warnt vor Fahrzeugen im toten Winkel, wenn es mehrere Fahrstreifen für die gleiche Fahrtrichtung gibt. Systeme dieser Art sind nach ISO 17387 eine Teilmenge der Fahrstreifenwechselassistenten. \footcite[Vgl.][S. 962]{Winner2015Fahrerassistenz} Zusätzliche Elemente, die ein Fahrstreifenwechselassistent besitzen kann, sind Warnungen vor weiter entfernten Fahrzeugen, die sich dem toten Winkel annähern. 
			\item Systeme, die beim Abbiegen vor Radfahrern oder Fußgängern mit Vorrang warnen, werden auch als Abbiegeassistenten bezeichnet. Sie bilden die zweite Gruppe von Totwinkelassistenten in dieser Arbeit. Vor allem LKWs werden mit diesen Systemen ausgestattet, da der tote Winkel bei Ihnen deutlich größer ist.
		\end{enumerate}
		Die Sensorik zur Hinderniserkennung wird in Abschnitt \ref{section:Hinderniserkennung} erläutert. Warn- und Eingriffsverhalten werden in den Abschnitten \ref{section:PassivesVerhalten} und \ref{section:AktivesVerhalten} beschrieben.
		\subsection{Hinderniserkennung}\label{section:Hinderniserkennung}
			Zur Erfassung der toten Winkel kann entweder eine Art der Sensoren aus Sektion \ref{section:Umgebungserfassung} eingesetzt werden, oder eine Kombination mehrerer. 
			
		\subsection{Passives Verhalten}\label{section:PassivesVerhalten}
		\subsection{Aktives Verhalten}\label{section:AktivesVerhalten}
	\section{Kamera-Monitor-Systeme}
		\subsection{Erweiterung / Ersetzung von Fahrzeugspiegeln}
		\subsection{Positionierung der Monitore}
	\section{Vergleich}