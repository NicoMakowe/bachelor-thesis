\chapter{Theoretische Grundlagen}
\fancyhfStyleContent{}


\section{Digitale Zwillinge}
\section{Aspektmodelle}
\section{Resource Description Framework (RDF)}

Das Resource Description Framework (RDF) ist ein Modell, das zur Beschreibung von Daten eingesetzt wird \cite[vgl.][]{w3c2014rdf}. Die Datenstruktur formt einen gerichteten Graphen, der sich aus Knoten und Verbindungen (Kanten) zwischen der Knoten zusammensetzt. Ein Ziel von RDF ist, einen Standard schaffen, der emöglicht, beliebige Informationen in maschinenlesbarer Form darzustellen \cite[vgl.][Sektion 2]{w3c2014rdfprimer}.

Zu RDF gehören mehrere Spezifikationen, die beispielsweise das Datenmodell oder eine Syntax zur Beschreibung von Daten festlegen. Das World Wide Web Consortium (W3O) veröffentlicht diese Spezifikationen und viele weitere unter dem Begriff "`Semantic Web"'. Dieser Ausdruck beschreibt die Vision, dass beliebige Daten im Internet bereitgestellt, ausgetauscht und verarbeitet werden können. \cite[vgl.][]{w3c2014semanticweb}

\subsection{Datenmodell}

Ein RDF-Graph setzt sich aus einer beliebig großen Menge sogenannter Triples zusammen. \cite[vgl.][Sektion 3.1]{w3c2014rdfconcepts} Ein einzelnes Triple ist formal gesehen ein 3-Tupel mit folgenden Elementen:
\begin{enumerate}
	\item Das erste Element bezeichnet man als Subjekt. Es repräsentiert den Startknoten, von dem eine Verbindung ausgeht.
	\item Das zweite Element nennt man Prädikat. Es stellt die konkrete Verbindung dar.
	\item Das dritte Element bezeichnet man als Objekt. Dies ist der Endknoten, auf den die Verbindung zeigt. 
\end{enumerate}

Somit sind Subjekt und Objekt immer ein Knoten, das Prädikat ist immer eine Kante.

Es gibt drei Arten von Knoten:
\begin{enumerate}
	\item Als Resource bezeichnet man einen Knoten, der sowohl Ein- als auch Ausgangsknoten besitzen kann und durch einen URI identifiziert.
	\item Blank Nodes sind Knoten, die Ein- und Ausgangsknoten besitzen können, aber nicht durch einen URI identifiziert sind.
	\item Literale sind Werte wie Ganzzahlen oder Strings und haben keine ausgehenden Kanten.	
\end{enumerate}
Kanten werden immer durch einen URI identifiziert.

In Tabelle \ref{tab:triples} sind Beispiele für Triples zu sehen, die zusammen einen Graphen formen. Die beiden URIs \textless urn:relation\#Anton\textgreater und \textless urn:relation\#Berta\textgreater sind Resourcen. \_:Stadt ist eine Blank Node und die Werte 34, "`Anton"' und "`Stuttgart"' sind Literale.

\begin{table}
	\centering
	\begin{tabular}{c|c|c}
		Subjekt & Prädikat & Objekt \\ \hline
		\textless urn:relation\#Anton\textgreater & \textless urn:relation\#name\textgreater & "'Anton"' \\
		\textless urn:relation\#Anton\textgreater & \textless urn:relation\#hatKind\textgreater & \textless urn:relation\#Berta\textgreater \\
		\textless urn:relation\#Berta\textgreater & \textless urn:relation\#hatAlter\textgreater & 34 \\
		\textless urn:relation\#Berta\textgreater & \textless urn:relation\#wohntIn\textgreater & \_:Stadt \\
		\_:Stadt & \textless urn:relation\#name\textgreater & "'Stuttgart"' \\
	\end{tabular}
	\caption{Beispiele für Triples}
	\label{tab:triples}
\end{table}
\subsection{RDF Syntax}

N-Triples ist eine Syntax, bei der beliebig viele Triples hintereinander aufgelistet werden \cite[vgl.][]{w3c2014ntriples}. Die Grammatik ist in Form von regulären Ausdrücken definiert. Ein Ausschnitt davon ist in den Formeln \ref{eq:ntriple} zu sehen. Jedes n-Triples-Dokument setzt sich demnach aus abwechselnden Triples und Zeilenumbrüchen (EOL, End-Of-Line) zusammen. Ein Triple besteht aus Subjekt, Prädikat und Objekt sowie einem Punkt am Ende. N-Triples-Dateien werden mit der Dateiendung "`.nt"' versehen. Die beispielhaften Triples aus Tabelle \ref{tab:triples} sind in Quellcode \ref{lst:n-triple} dargestellt.
\begin{align}
	\text{ntripleDoc} & ::= \text{triple?}\text{ } (\text{EOL triple})\mbox{*} \text{ EOL?} \label{eq:ntriple}\\
	\text{triple} & ::= \text{subject } \text{predicate } \text{object } \text{"'."'} \nonumber
\end{align}
\listingfile[caption=Beispiel einer N-Triples-Datei, label=lst:n-triple]{resources/codeSnippets/2PureTriples.ttl}

Eine zweite Syntax, mit der RDF-Graphen beschrieben werden können, heißt "`Terse RDF Triple Language"' (TTL) und wird aufgrund der  auch als "`Turtle"' bezeichnet  \cite[vgl.][]{w3c2014turtle}. Turtle-Dateien besitzen die Datei-Endung "`.ttl"'.
Die Sprache, die von der TTL-Grammatik beschrieben wird, ist eine Obermenge der N-Triples-Sprache. Das heißt, jede gültige N-Triples-Datei ist auch eine gültige Turtle-Datei. Zusätzlich erlaubt die Turtle-Syntax weitere Schreibweisen, die es ermöglichen, Graphen übersichtlicher und mit weniger Text darzustellen. Durch syntaktische Äquivalenzumformungen lässt sich jede Turtle-Datei wieder auf eine Liste von Triples zurückführen. Somit kann jeder beliebige Graph sowohl mit N-Triples als auch mit Turtle beschrieben werden.

Im Folgenden werden einige wichtige Syntax-Merkmale genauer vorgestellt.
\paragraph{Namespaces}

Die URIs in einer Datei beginnen häufig mit dem gleichen Zeichenfolge. Benannte Knoten unterscheiden sich meist nur am letzten Abschnitt des URIs. Namespaces ermögichen, eine abgekürzte Schreibweise für URIs zu verwenden, um die Datei übersichtlicher zu machen. Ein Namespace wird mit dem Schlüsselwort \lstinline|@prefix| festgelegt, wie in Quellcode \ref{lst:ttl-namespace} zu sehen ist.

\listingfile[caption=TTL-Datei mit Namespace, label=lst:ttl-namespace]{resources/codeSnippets/2TriplesNamespace.ttl}

\paragraph{Liste von Prädikaten}


\section{Software Defined Vehicle}

