\chapter{Grundlagen}
	\section{Toter Winkel}
		Ein toter Winkel im Straßenverkehr ist ein Bereich, den der Fahrer durch ein nach vorne gerichtetes Sichtfeld weder direkt noch durch einen Spiegel einsehen kann. Für diese Arbeit wird zwischen zwei Typen von toten Winkeln unterschieden.
		
		Ein toter Winkel, der durch die A-Säule verursacht wird, verläuft von der linken bzw. rechten A-Säule kegelförmig nach vorne links bzw. vorne rechts. Die A-Säule verläuft von der Motorhaube bis zum Dach und ist die Trennung zwischen Frontscheibe und Fahrer- bzw. Beifahrertür. Tote Winkel dieser Art sind bei kleinen Entfernungen schmal sodass ein selbst Fahrrad, das in unmittelbarer Nähe zum Fahrzeug ist, nicht vollständig in den toten Winkel passt. Folgende Rechnung veranschaulicht dies.
		Die linke A-Säule sei ein senkrechter Zylinder mit einem Durchmesser von $s_{A}=\SI{0,10}{m}$. Die Entfernung zwischen Fahrer und A-Säule betrage $d_{AD}=\SI{1,0}{m}$ und die Breite eines Fahrrad sei $s_{B}=\SI{1.0}{m}$. Mithilfe eines Strahlensatzes lässt sich herleiten, dass das Fahrrad mindestens $\SI{10}{m}$ vom Fahrer entfernt sein muss, dass der Fahrer das Fahrrad nicht mehr einsehen kann.
		\begin{align*}
			s_{A}&=\SI{0,10}{m} & &d_{AD}=\SI{1,0}{m} && s_{B}=\SI{1.0}{m} && \frac{d_{AD}}{s_A}=\frac{d_{BD}}{s_{B}} \Rightarrow d_{BD}=\SI{10}{m}
		\end{align*}
		Weiterhin gilt, dass die A-Säule nicht genau senkrecht sondern schräg verläuft und dass ein gesunder Fahrer mit beiden Augen schaut und sich somit der tote Winkel verkleinert. Auch wenn sich das Fahrrad für einen Moment im toten Winkel befindet, wird es aufgrund kleiner Geschwindkeitsdifferenzen zwischen Fahrzeug und Fahrrad kurze Zeit später wieder sichtbar sein. Deshalb geht von dieser Art des toten Winkels ein geringes Gefahrenpotenzial aus.
		
		Die Bereiche hinter dem Fahrer, die nicht von den Außenspiegeln eingesehen werden können, bilden den zweiten Typ von toten Winkeln. Diese Bereiche sind wesentlich breiter und sodass ein anderer Verkehrsteilnehmer leichter in diesen Bereich kommt.
		
		Bei kleinen Fahrzeugen kann ein großer Teil dieser toten Winkel durch einen Schulterblick eingesehen werden. Bei größeren Fahrzeugen wie Vans oder LKWs ist dies nur für kleine Teile des toten Winkels möglich. Gefährliche Situationen sind beispielsweise Spurwechseln bei dichtem Verkehr in der Stadt oder auf der Autobahn sowie Rechtsabbiegen im Stadtverkehr, wenn Fußgänger oder Fahrräder Vorrang haben.
	\section{Umgebungserfassung}\label{section:Umgebungserfassung}
		\subsection{Ultraschall}
			Ultraschallsensoren zur Messung von Distanzen werden im Automotive-Bereich für Einparksensoren eingesetzt. \footcite[Vgl.][]{ChenXiao2009Parking} Zur Messung werden Schallwellen ausgesendet, die außerhalb des Frequenzbereich des menschlichen Ohrs liegen. Trifft die Schallwelle auf ein Objekt, wird sie reflektiert und kommt zum Sender zurück. Durch den piezoelektrischen Effekt wird eine elektrische Spannung erzeugt. Ein Steuergerät, das mit dem Sensor verbunden ist, kann anhand der Zeitspanne zwischen Senden und Empfangen, die Entfernung zum Objekt berechnen. \footcite[Vgl.][S. 244, S. 252]{Winner2015Fahrerassistenz}
		
			Die Vorteile von ultraschallbasierter Entfernungsmessung sind, dass die Hardware kostengünstig ist und dass die Messung sehr fehlertolerant ist. Das ist unter anderem darauf zurückzuführen, dass die Schallgeschwindigkeit (im Vergleich zur Lichtgeschwindigkeit) gering ist. Ein Nachteil, Ein Nachteil ist, dass schallabsorbierende Stoffe wie Schaumstoff oder Fell schlechter detektiert werden können. \footcite[Vgl.][S. 256f]{Winner2015Fahrerassistenz} 
			Enfernungsmessungen basierend auf Ultraschall sind ungeeignet für große Distanzen, da das Signal durch Dämpfung schwächer wird und es durch die langsame Schallgeschwindigkeit zu Verzögerungen kommen kann. Die Maximalentfernung ist vom Hersteller typischerweise im Meter- bis 10-Meter-Bereich angegeben. \footcite[Vgl.][]{Sparkfun2022Sensor}, \footcite[Vgl.][]{APG2022Sensor}
		\subsection{Elektromagnetische Wellen}
			In der Praxis werden zwei Systems zur Umgebungserfassung verwendet, deren Messung auf elektromagnetische Wellen basiert: das Radar und das Lidar.
			
			\textbf{Radar} steht für "radio detection and ranging" und arbeitet mit elektromagnetischen Wellen im Millimeterbereich. Ähnlich wie beim Einsatz von Ultraschall werden Wellen ausgesendet und an Objekten reflektiert. \footcite[Vgl.][S. 260, S. 264-267]{Winner2015Fahrerassistenz} Mithilfe des reflektierten Signals kann die Entfernung zum Objekt berechnet werden. Es gibt verschiedene Umsetzungen. Ein Impulsradar sendet kurze Pulse aus und wartet auf das reflektierte Signal. Ein Dauerstrichradar (FMCW-Radar) sendet dauerhaft ein frequenz- oder amplitudenmoduliertes Signal aus und errechnet die Verzögerung mithilfe einer Demodulation.
			
			Mit einem Radar kann nicht nur die Entfernung gemessen werden, sondern auch die relative Geschwindigkeit zu einem anderen Objekt. Diese Messung basiert auf den Dopplereffekt und ermöglicht, dass die Geschwindigkeit präzise gemessen werden kann, ohne dass die Geschwindigkeit aus mehreren Entfernungsmessungen errechnet werden muss. \footcite[Vgl.][]{Xu2014Doppler}
			
			\textbf{Lidar} steht für "Light Detection And Ranging" und nutzt im Gegensatz zum Radar elektromagnestische Wellen im Frequenzbereich des sichtbaren Lichts bzw. des Infrarot- oder Ultraviolett-Bereichs. \footcite[Vgl.][S. 318]{Winner2015Fahrerassistenz} Auch beim Lidar wird die Verzögerung zwischen Aussenden des Signals und Empfangen des reflektierten Signals zur Entfernung genutzt. Man bezeichnet diese Messung als Time-Of-Flight-Messung. \footcite[Vgl.][]{Ewald2000LaserObstacle}
			
			Da Laserstrahlen sich nicht radial oder kegelförmig ausbreiten, sondern annähernd auf einer Geraden, wird ein rotierender Spiegel eingesetzt, um einen breiteren Bereich abzudecken. Wird dieser zusätzlich gekippt, können Messungen im dreidimensionalen Raum durchgeführt werden. \footcite[Vgl.][]{Zhaohua2020RadarLidar}
			
			Radar und Lidar haben beide Vor- und Nachteile. Das Lidar kann Entfernungen und Positionen im dreidimensionalen Raum genau messen, aber wird durch bestimmte Wetterbedingungen wie Regen und Schnee stärker beeinträchtigt als das Radar. Beim Radar kann aufgrund der größeren Wellenlänge der Dopplereffekt besser zur Geschwindigkeitsmessung genutzt werden. Auch die Reichweite ist beim Radar mit 200 Metern größer als beim Lidar (120 Meter) \footcite[Vgl.][]{Zhaohua2020RadarLidar}
		\subsection{Kamera}
		Kamerasysteme sind eine weitere Option, um andere Fahrzeuge und Hindernisse zu detektieren. Sie sind kostengünstiger als Radar- und Lidar-Systeme, aber haben eine geringere Reichweite. Ein Stereo-Kamera-System ist in der Lage, Entfernungen von Objekten abzuschätzen. \footcite[Vgl.][]{Tsai2018StereoVision} Ein Nachteil von Kameras ist, dass die Messung bei bestimmten Lichtverhältnissen (zum Beispiel Nacht oder Nebel) deutlich eingeschränkt ist.
	\section{Fahrerwarnung}
		Es gibt verschiedene Möglichkeiten, den Fahrer in einer Gefahrensituation zu warnen. Drei Klassen von Warnungen werden im Folgenden vorgestellt.
		\begin{itemize}
			\item Bei optischen Warnungen wird der Fahrer über eine Kontrollleuchte oder eine Information auf dem Bordcomputer informiert. Fahrzeuge, die über ein Head-Up-Display verfügen, können Warnungen auf die Frontscheibe projizieren und Reaktionszeit bei Gefahren (im Vergleich zu keiner Warnung) verringern.
			\footcite[Vgl.][]{Kazazi2015HeadUpDisplay}
			\item Akustische Warnungen werden meist bei akuter Gefahr eingesetzt, da sie den Fahrer unabhängig von der Blickrichtung warnen können. Eine Studie von 2008 zeigt, dass Fahrer verschiedene akustische Warnungen besser unterscheiden kann, wenn die Warnsignale über unterschiedliche Lautsprecher abgespielt werden. \footcite[Vgl.][]{Sato2008SoundWarnings} Daraus lässt sich vermuten, dass es hilfreich ist, bei Gefahr von der Seite, einen Warnton nur auf den Lautsprechern dieser Seite abzuspielen.
			\item Zur Gruppe der haptische Warnungen gehören Warnungen, die der Fahrer mit dem Tastsinn wahrnimmt, zum Beispiel Vibrationen oder Kraftausübung auf das Lenkrad in Form von Gegenlenken. Die Trennung zwischen Warnung und Eingriff ist hierbei unscharf. Studien legen nahe, dass haptische Warnungen die Reaktionszeit verringern können. \footcite[Vgl.][]{Petermeijer2015Haptic}
		\end{itemize}
