\begin{titlepage}
	\begin{center}
		\begin{textblock*}{210mm}(-2.5cm, -3.2cm)
			\includegraphics[width=\paperwidth, height=4mm]{images/shared/Bosch-Supergraphic-Crop}
		\end{textblock*}
	\end{center}

	\vspace{-2.0cm}
	\begin{table}[h]
		\centering
		\begin{tabular}{p{0.45\textwidth}p{0.05\textwidth}p{0.45\textwidth}}
				\includegraphics[width=0.45\textwidth]{images/shared/Bosch-Logo-Ohne-Supergraphic} & &
				\raggedright
				\includegraphics[width=0.4\textwidth]{images/shared/DHBW-Logo}
		\end{tabular}
	\end{table}
		
	\begin{center}
		\LARGE
		\vspace*{1.5cm}
		
		\textbf{Entwicklung einer Schnittstelle zur Verwaltung von Fahrer- und Beifahrersitzfunktionalität unter Zuhilfenahme Digitaler Zwillinge}
		
		\vspace{1cm}
		\Large
		Bachelorarbeit
	
		\large
		\vspace{1cm}
		des Studiengangs Informatik \\ an der Dualen Hochschule Baden-Würrtemberg Stuttgart
		
		\vspace{1cm}
		von
		
		\Large
		\vspace{1cm}
		Nico Makowe
		
		\vspace{1cm}
		\today
		
		\vfill
		\large
		\begin{tabular}{l l}
			Bearbeitungszeitraum: & 12 Wochen \\
			Matrikelnummer, Kurs: & 9275184, STG-TINF19ITA \\
			Dualer Partner: & Robert Bosch GmbH \\
			Betreuer des Dualen Partners: & Georg Schmidt-Dumont\\
			Gutachter der Dualen Hochschule: & Jamal Krini
		\end{tabular}
	\end{center}
\end{titlepage}